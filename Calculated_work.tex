\documentclass[14pt, a4paper, ukrainian]{article}

\usepackage[14pt]{extsizes}
\usepackage{cmap}
\usepackage[utf8]{inputenc}
\usepackage[T2A]{fontenc}
\usepackage[english, ukrainian]{babel}

\title{Розрахункова робота № 1}
\author{Олександр Вергелюк}
\date{\today}


\usepackage[left=3.00cm, right=1.50cm, top=2.00cm, bottom=2.00cm]{geometry}

%Робота з математикою 
\usepackage{graphicx}
\usepackage{amsmath, amsfonts, amssymb, mathtools} %AMS
\usepackage{icomma} %Розумна кома
\usepackage{indentfirst}
\usepackage[usenames,dvipsnames]{color}
\usepackage{makecell}
\usepackage{multirow}
\usepackage{ulem}

%Шрифти
\usepackage{euscript}
\usepackage{mathrsfs}
\linespread{1.3} % полуторный интервал

%Власні команди
\DeclareMathOperator{\sgn}{mathop{sgn}}

%Перенесення знаків у формулах (За Львовським)
\newcommand*{\hm}[1]{#1\nobreak\discretionary{}%
	{\hbox{$\mathsurround=0pt #1$}}{}}

\begin{document}
\begin{titlepage}
	\centering
	\vspace{1cm}
	{ МІНІСТЕРСТВО ОСВІТИ І НАУКИ УКРАЇНИ\\
		НАВЧАЛЬНО-НАУКОВИЙ КОМПЛЕКС\\
		``ІНСТИТУТ ПРИКЛАДНОГО СИСТЕМНОГО АНАЛІЗУ``\\
		НАЦІОНАЛЬНОГО ТЕХНІЧНОГО УНІВЕРСИТЕТУ УКРАЇНИ\\
		``КИЇВСЬКИЙ ПОЛІТЕХНІЧНИЙ ІНСТИТУТ ІМЕНІ ІГОРЯ СІКОРСЬКОГО``\\
		КАФЕДРА МАТЕМАТИЧНИХ МЕТОДІВ  СИСТЕМНОГО АНАЛІЗУ\\\par}
	\vspace{5cm}
\Large \textsc{\textbf{розрахункова робота №1}}\\
\large {з теорії ймовірності} \\
на тему: {<<Випадкові вектори>>}
\vfill
\newlength{\ML}
\settowidth{\ML}{\hspace{3.4cm}}
\hfill
\begin{minipage}{0.4\textwidth}
	Виконав студент 2 курсу групи КА-06\\
	Вергелюк Олександр Андрійович
	
	Перевірив: \\
	Ільєнко А. Б.
\end{minipage}
\vfill
\begin{center}
Київ -- 2021
\end{center}
\end{titlepage}
	
	
\end{document}