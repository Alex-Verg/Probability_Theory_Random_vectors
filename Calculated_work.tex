\documentclass[14pt, a4paper, ukrainian]{extreport}

\usepackage[14pt]{extsizes}
\usepackage{cmap}
\usepackage[utf8]{inputenc}
\usepackage[T2A]{fontenc}
\usepackage[english, ukrainian]{babel}
\usepackage{slashbox}
\usepackage{caption}
\DeclareCaptionLabelFormat{gostfigure}{Рисунок #2}
\DeclareCaptionLabelFormat{gosttable}{Таблиця #2}
\DeclareCaptionLabelSeparator{gost}{~---~}
\captionsetup{labelsep=gost}
\captionsetup*[figure]{labelformat=gostfigure}
\captionsetup*[table]{labelformat=gosttable}
\captionsetup*[figure]{labelformat=gostfigure, justification=centering}  % выравнивание по центру



\usepackage{titlesec}

\titleformat{\chapter}[block]
{\filcenter}
{\thechapter}
{1em}
{\MakeUppercase}
{}

\titlespacing*{\chapter}{0pt}{-40pt}{*4} 

\titleformat{\section}
{\filright}
{\thesection}
{1ex}{}

\titlespacing*{\section}{\parindent}{*6}{*6}


\title{Розрахункова робота № 1}
\author{Олександр Вергелюк}
\date{\today}


\usepackage[left=3.00cm, right=1.50cm, top=2.00cm, bottom=2.00cm]{geometry}

%Робота з математикою 
\usepackage{graphicx}
\usepackage{amsmath, amsfonts, amssymb, mathtools} %AMS
\usepackage{icomma} %Розумна кома
\usepackage{indentfirst}
\parindent 1.25cm
\usepackage[usenames,dvipsnames]{color}
\usepackage{makecell}
\usepackage{multirow}
\usepackage{ulem}
\usepackage{float}

%Шрифти
\usepackage{euscript}
\usepackage{mathrsfs}
\linespread{1.3} % полуторный интервал

%Власні команди
\DeclareMathOperator{\sgn}{mathop{sgn}}

%Перенесення знаків у формулах (За Львовським)
\newcommand*{\hm}[1]{#1\nobreak\discretionary{}%
	{\hbox{$\mathsurround=0pt #1$}}{}}

\begin{document}
\begin{titlepage}
	\centering
	\vspace{1cm}
	{ МІНІСТЕРСТВО ОСВІТИ І НАУКИ УКРАЇНИ\\
		НАВЧАЛЬНО-НАУКОВИЙ КОМПЛЕКС\\
		``ІНСТИТУТ ПРИКЛАДНОГО СИСТЕМНОГО АНАЛІЗУ``\\
		НАЦІОНАЛЬНОГО ТЕХНІЧНОГО УНІВЕРСИТЕТУ УКРАЇНИ\\
		``КИЇВСЬКИЙ ПОЛІТЕХНІЧНИЙ ІНСТИТУТ ІМЕНІ ІГОРЯ СІКОРСЬКОГО``\\
		КАФЕДРА МАТЕМАТИЧНИХ МЕТОДІВ  СИСТЕМНОГО АНАЛІЗУ\\\par}
	\vspace{5cm}
\Large \MakeUppercase {\textsc{\textbf{{розрахункова робота №1}}}}\\
\Large {з теорії ймовірності} \\
на тему: {<<Випадкові вектори>>}\\
Варіант 7
\vfill
\newlength{\ML}
\settowidth{\ML}{\hspace{3.4cm}}
\hfill
\begin{minipage}{0.4\textwidth}
	Виконав студент 2 курсу групи КА-06\\
	Вергелюк Олександр Андрійович
	
	Перевірив: \\
	Ільєнко А. Б.
\end{minipage}
\vfill
\begin{center}
Київ -- 2021
\end{center}
\end{titlepage}
	\chapter{Завдання 1}
	
	Нехай дискретний випадковий вектор $\vec\xi = (\xi_1, \xi_2)$ задано таблицею розподілу (Таблиця {\ref {tab:task1}}).
	
	\begin{table}[H]
		\caption {\label{tab:task1} Таблиця розподілу вектора $\vec \xi$}
		
		\begin{center}
			\resizebox*{!}{!}{
				\begin{tabular}{ | c | c | c | c | c | }
					\hline
					\backslashbox {$\xi_1$}{$\xi_2$} & -9 & 2 & 7 & 8\\
					\hline
					-2 & 0.17 & 0.01 & 0.07 & 0.06 \\
					\hline
					1 & 0.13 & 0.08 & 0.04 & 0.08 \\
					\hline
					2 & 0.12 & 0.02 & 0.07 & 0.15 \\
					\hline
				\end{tabular}
			}
		\end{center}
	\end{table}
	
	\section{Ряди розподілу координат $\xi_1$ та $\xi_2$}
	
	Для $\xi_1$ використаємо формулу: $ p_i = \mathbb{P} \{\xi_1 = x_i\} = \sum_{j = 1}^{n} p_{ij}$.
	
	\begin{table}[H]
		\caption{\label{tab:xi1} Ряд розподілу $\xi_1$}
		\begin{center}
			\resizebox*{!}{!}{
				\begin{tabular}{| c | c | c | c |}
					\hline
					$x_i$ & -2 & 1 & 2 \\
					\hline
					$p_i$ & 0.31 & 0.33 & 0.36 \\
					\hline
				\end{tabular}
		}
		\end{center}
	\end{table}
	
	Перевірка: 	$$ 0.31 + 0.33 + 0.36 = 1$$	
	Аналогічно для $\xi_2$: $ p_j = \mathbb{P} \{\xi_2 = y_j\} = \sum_{i = 1}^{m} p_{ij} $.
	
	\begin{table}[H]
		\caption{\label{tab:xi2} Ряд розподілу $\xi_2$}
		\begin{center}
			\resizebox*{!}{!}{
			\begin{tabular}{| c | c | c | c | c |}
				\hline
				$y_i$ & -9 & 2 & 7 & 8\\
				\hline
				$p_i$ & 0.42 & 0.11 & 0.18 & 0.29\\
				\hline
			\end{tabular}
		}
		\end{center}
	\end{table}

	Перевірка:  $$ 0.42 + 0.11 + 0.18 + 0.29 = 1 $$

	\section{Функції розподілу $F_{\xi_1}(x)$ і $F_{\xi_2}(y)$ та їх графіки}
	
	Для координати $\xi_1$:
	
	$$ F_{\xi_1} =  \begin{cases}
		0  & x \le -2\\
		0.31 & -2 < x \le 1\\
		0.64 & 1 < x \le 2\\
		1 & 2 < x
	\end{cases}
	$$
	
	\begin{figure}[H]
		\centering
		\includegraphics[width=\textwidth]{./Image/Im_01_Fx.png}
		\caption{Функція розподілу $F_{\xi_1}(x)$}
		\label{im:Fx}
	\end{figure}

	Для координати $\xi_2$:
	
	$$F_{\xi_2}(y) = \begin{cases}
		0 & y \le -9\\
		0.42 & -9 < y \le 2\\
		0.53 & 2 < y \le 7\\
		0.71 & 7 < y \le 8\\
		1 & 8 < y
	\end{cases}
	$$

	\begin{figure}[H]
		\centering
		\includegraphics[width=\textwidth]{./Image/Im_02_Fy.png}
		\caption{Функція розподілу $F_{\xi_2}(y)$}
		\label{im:Fy}
	\end{figure}
	
	
	\section{Функція розподілу $F_{\vec \xi}(x, y)$ випадкового вектора}
	
	Зобразимо в декартовій системі координат всі точки, що відповідають значенню вектора $\vec \xi$ (рис. \ref{im:Fxy}).

	\begin{figure}[h!]
		\centering
		\includegraphics[width=\textwidth]{./Image/Im_03_Fxy_1}
		\caption{Значення вектора $\vec \xi$ в декартовій системі координат}
		\label{im:Fxy}
	\end{figure}

	Розіб'ємо координатну площину на області, в яких сумісна функція розподілу $F_{\vec \xi}(x, y)$ набуває однакові значення (рис. \ref{im:Fxy_D}).
		
	\begin{figure}[h!]
		\centering
		\includegraphics[width=\textwidth]{./Image/Im_04_Fxy_2}
		\caption{Області, в яких сумісна функція розподілу $F_{\vec \xi}(x, y)$ набуває однакові значення}
		\label{im:Fxy_D}
	\end{figure}

	Використаємо формулу:
	$$ F_{\vec \xi }(x, y) = \mathbb{P} \{\xi_1 < x, \xi_2 < y\} = \sum_{x_i < x} \sum_{y_j = y} p_{ij}$$
	a) $(x, y) \in D_0 \Rightarrow F_{\vec \xi}(x, y) = 0$\\
	b) $(x, y) \in D_1 \Rightarrow F_{\vec \xi}(x, y) = 0.17$\\
	c) $(x, y) \in D_2 \Rightarrow F_{\vec \xi}(x, y) = 0.3$\\
	d) $(x, y) \in D_3 \Rightarrow F_{\vec \xi}(x, y) = 0.42$\\
	e) $(x, y) \in D_4 \Rightarrow F_{\vec \xi}(x, y) = 0.18$\\
	f) $(x, y) \in D_5 \Rightarrow F_{\vec \xi}(x, y) = 0.39$\\
	g) $(x, y) \in D_6 \Rightarrow F_{\vec \xi}(x, y) = 0.53$\\
	h) $(x, y) \in D_7 \Rightarrow F_{\vec \xi}(x, y) = 0.25$\\
	i) $(x, y) \in D_8 \Rightarrow F_{\vec \xi}(x, y) = 0.5$\\
	j) $(x, y) \in D_9 \Rightarrow F_{\vec \xi}(x, y) = 0.71$\\
	k) $(x, y) \in D_10 \Rightarrow F_{\vec \xi}(x, y) = 0.31$\\
	l) $(x, y) \in D_11 \Rightarrow F_{\vec \xi}(x, y) = 0.64$\\
	m) $(x, y) \in D_12 \Rightarrow F_{\vec \xi}(x, y) = 1$\\
	
	Запишемо функцію розподілу у вигляді таблиці:
	
	\begin{table}[H]
		\caption{\label{tab:Fxy}Сумісна функція розподілу $F_{\vec \xi}(x, y)$}
		\begin{center}
			\begin{tabular}{| c | c | c | c | c | c |}
				\hline
				\backslashbox{$x$} {$y$} & $y \le -9$ & $ -9 < y \le 2$ & $ 2 < y \le 7 $& $ 7 < y \le 8 $ & 8 < y\\
				\hline
				$ x \le -2 $ & 0 & 0 & 0 & 0 & 0\\
				\hline
				$ -2 < x \le 1 $ & 0 & 0.17 & 0.18 & 0.25 & 0.31 \\
				\hline
				$ 1 < x \le 2 $ & 0 & 0.3 & 0.39 &  0.5 & 0.64\\
				\hline
				$ 2 < x $ & 0 & 0.42 & 0.53 & 0.71 & 1\\
				\hline
				
			\end{tabular}
		\end{center}
	\end{table}

	Перевірка (властивість узгодження):\\
	$$ \lim\limits_{y \to \infty} F_{\vec \xi} (x,y) = F_{\xi_1} (x) \text{ виконується, оскільки останній стовпчик -- це } F_{\xi_1} (x)$$
	 $$\lim\limits_{x \to \infty} F_{\vec \xi} (x,y) = F_{\xi_2} (y) \text{ виконується, оскільки останній рядок -- це } F_{\xi_2} (y)$$
	 $$\lim\limits_{
		\begin{gathered}
			x \to \infty\\
			y \to \infty
		\end{gathered}
		} 
		F_{\vec \xi} (x,y) = 1  \text{ виконується, оскільки } (x, y) \in D_12 \Rightarrow F_{\vec \xi} (x, y) = 1 $$
	
	\section{Математичні сподівання координат та кореляційна матриця}
	
	Знайдемо математичне сподівання координати $\xi_1$:
	$$ \mathbb{E}\xi_1 = \sum_{i=1}^{3} x_ip_i = (-2) \cdot 0.31 + 1 \cdot 0.33 + 2 \cdot 0.36 = 0.43 $$
	
	Аналогічно для координати $\xi_2$:
	$$ \mathbb{E}\xi_2 = \sum_{j=1}^{4} x_jp_j = (-9) \cdot 0.42 + 2 \cdot 0.11 + 7 \cdot 0.18 + 8 \cdot 0.29 = 0.02$$
	
	Центр розсіювання вектора $\vec \xi$ -- точка (0.43, 0.02).
	
	Дисперсія координати $\xi_1$:
	$$ \mathbb{D}\xi_1 = \mathbb{E}({\xi_1 - \mathbb{E}\xi_1})^2 = \mathbb{E}\xi_1^2 - (\mathbb{E}\xi_1)^2 = (-2)^2 \cdot 0.31 + 1^2 \cdot 0.33 + 2^2 \cdot 0.36 = 3.01 $$
	
	Дисперсія координати $\xi_2$:	
	$$\mathbb{D}\xi_2 = \mathbb{E}({\xi_2 - \mathbb{E}\xi_2})^2 = \mathbb{E}\xi_2^2 - (\mathbb{E}\xi_2)^2 =		
		(-9)^2 \cdot 0.42 + 2^2 \cdot 0.11 + 7^2 \cdot 0.18 + 8^2 \cdot 0.29 = 61.84	$$
			
	Для побудови коваріаційної матриці скористаємося формулами:\\
		$cov(\xi_1\xi_2) = \mathbb{E}\xi_1\xi_2 - \mathbb{E}\xi_1\mathbb{E}\xi_2\\
		cov(\xi_1\xi_1) = \mathbb{D}\xi_1\\
		cov(\xi_2\xi_2) = \mathbb{D}\xi_2$
		
	$$\mathbb{E}\xi_1\xi_2 = \sum_{i=1}^{3}\sum_{j=1}^{4}x_iy_jp_{ij} = (-2)\cdot(-9)\cdot0.17 + (-2)\cdot2\cdot0.01 + $$
	$$ (-2)\cdot7\cdot0.07 + (-2)\cdot8\cdot0.06 + 1\cdot(-9)\cdot0.13 + 1\cdot2\cdot0.08 +$$
	$$ + 1\cdot7\cdot0.04 + 1\cdot8\cdot0.08 + 2\cdot(-9)\cdot0.12 + 2\cdot2\cdot0.02 + $$
	$$ +2\cdot7\cdot0.07 + 2\cdot8\cdot0.15 = $$
	
	$$ cov(\xi_1\xi_2) =  - 0.43\cdot0.02 = $$
	
	Тоді коваріаційна матриця:
	
	$$ C{\vec\xi} = \begin{pmatrix}
		\mathbb{D}\xi_1 & cov(\xi_1\xi_2)\\
		cov(\xi_1\xi_2) & \mathbb{D}\xi_1\\
	\end{pmatrix} = 
	\begin{pmatrix}
		3.01 & \\
		& 61.84\\
	\end{pmatrix}
	$$
	
	Оскільки, $ \det{C\vec{\xi}} \neq 0$, то випадкові величини $\xi_1$ та $\xi_2$ корельовані та залежні. 
	\section{Умовні ряди розподілу}

\end{document}